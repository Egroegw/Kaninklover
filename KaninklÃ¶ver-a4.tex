\documentclass[a4paper, twoside, 11pt]{article}
% \documentclass[twoside, 11pt]{article}
\usepackage[utf8]{inputenc}
\usepackage[inner=0.6in, outer=0.6in, top=0in, bottom=1.0in]{geometry}
\usepackage{titlesec}
\usepackage{multicol}
\usepackage{threeparttable}
\usepackage{parskip}
\usepackage{enumitem}
\usepackage{etoolbox}
\usepackage{setspace}
\usepackage{fancyhdr}
\usepackage{lastpage}
\usepackage{graphicx}
% \usepackage{ulem}
\usepackage[sfdefault]{atkinson}

% \newgeometry{inner=0.4in, outer=0.4in, top=1.2in, bottom=1.0in}
\newgeometry{inner=0.6in, outer=0.6in, top=1.2in, bottom=1.0in}
\overfullrule=0pt
\setlist[itemize]{parsep=3pt}
\setlist[enumerate]{parsep=3pt}


% \titleformat{\subsection}{\bfseries\itshape\large}{\thesection.}{}{}
\linespread{1.0}
\setlength{\parindent}{0pt}
\setlength\columnsep{15pt}

\makeatletter
\patchcmd{\@maketitle}{\LARGE}{\Huge}{\typeout{OK 1}}{\typeout{Failed 1}}
\patchcmd{\@maketitle}{\large \lineskip}{\Large \lineskip}{\typeout{OK 2}}{\typeout{Failed 3}}
\patchcmd{\maketitle}{\@fnsymbol}{\@arabic}{}{}
\makeatother

\pagestyle{fancy}
\fancyhf{}
\fancyhead[LE,RO]{Wang (2021), \textit{Kaninklöver/Cottontail Club: A Cheatsheet}}
\fancyhead[RE,LO]{Larsson and Uus}
% \fancyfoot[CE,CO]{\leftmark}
\fancyfoot[C]{\thepage\ of \pageref{LastPage}}
% \fancyfoot[LE,RO]{\thepage}
\renewcommand{\headrulewidth}{2pt}
\renewcommand{\footrulewidth}{1pt}

\setlength{\skip\footins}{0.8cm}



% \renewcommand{\baselinestretch}{1.0}
% \usepackage[portrait, margin=0.6in]{geometry}
% \addtolength{\topmargin}{-.6in}
% \usepackage[colors=2]{onedown}
% \renewcommand{\familydefault}{\sfdefault}
\renewcommand{\thefootnote}{\arabic{footnote}}

\title{Kaninklöver / Cottontail Club\\A Cheatsheet (2021)}

\author{
Jan Eric Larsson\footnote{\normalsize{\texttt{ jan@goalart.com}}} \hphantom aand Anu Uus \\
Revision by George Wang\footnote{\normalsize{\texttt{ egroegw@tutanota.com.} \\
This document  in LaTeX and PDF form is available at \texttt{https://github.com/Egroegw/Kaninklover}}}
}
\date{}

\begin{document}
% \sffamily
\maketitle

\vspace{0cm}
% \hline
\begin{multicols}{3}

\itshape

Kaninklöver is a system developed by Larsson and Uus and descended from Larsson's earlier Tangerine/Cranberry Club.

While summarising, I consulted
\begin{itemize}
    \item \emph{The Tangerine Club,} pp. 2--20, Larsson 1995
    \item \emph{Kaninklöver,} pp. 3--38, Larsson 2015, and
    \item \emph{Kaninklöver Sammanfattning,} Larsson and Uus 2020
\end{itemize}

For more notes, see the last page.

\upshape
\section{Opening Bids}
In order of attractiveness,
\begin{enumerate}
    \item Open 1NT or 2NT if you can
    \item Open 1$\clubsuit$ if you can
    \item Open your \textit{longest suit}.
    In a tie, open the higher-ranking suit unless 44 in the majors

     \textbf{Exception:} With exactly 3334 shape, open 1$\diamondsuit$
    \item Preempt if you can/dare!
    \item Pass
\end{enumerate}



\section{1$\clubsuit$ opening}

15+ points, any distribution.


\subsection*{Negative response}

Responder's 1$\diamondsuit$ shows any 0--8 HCP.

\begin{center}
\begin{tabular}{ |c|l| }
 \hline
 \multicolumn{2}{|c|}{\textit{Opener's 2nd bid}} \\
 \hline
 1$\heartsuit\spadesuit$ & 15--20, 4+ cards\\
 1NT& balanced minimum \\
 2$\clubsuit\diamondsuit$ & 15--20, 5+ cards\\
 other& 21+, \textit{särskild Stödfrag}\\
 2NT& 23--24, balanced\\

 \hline
\end{tabular}
\end{center}

With 0--4 HCP, Responder passes.

Opener bids suits naturally; responder bids minors natually but bids [$\heartsuit, \spadesuit$, NT] to show [$\spadesuit$, NT, $\heartsuit$] respectively.

If "balanced minimum" for opener is 15--17, she shows a balanced 18--20 by bidding her longest suit (1$\heartsuit\spadesuit$, 2$\clubsuit\diamondsuit$), intending to rebid NT.\\



% \begin{tabular}{ |c|c|l|  }

%  \hline
%  \multicolumn{3}{|c|}{\textit{Opener's 2nd bid}} \\
%  \hline
%  1$\heartsuit\spadesuit$ & 15--20& 5-card suit\\
%  1NT& 18--19& balanced, no 5cM\\
%  2$\clubsuit\diamondsuit$ & 15--20& 5-card suit\\
%  2$\heartsuit\spadesuit$ & 21+& \textit{särskild Stödfrag}\\
%  3$\clubsuit\diamondsuit$ & 21+& \\
%  \hline
% \end{tabular}

\subsection*{Positive response: 1$\clubsuit$ -- other}

\begin{center}
\begin{tabular}{ |c|l| }
 \hline
 \multicolumn{2}{|c|}{\textit{Responder's 1st bid}} \\
 \hline
 1$\heartsuit$ & 10+, 5+ spades\\
 1$\spadesuit$ & 10+,  balanced\\
 1NT& 10+, 5+ hearts\\
 2$\clubsuit$ & 10+, 5+ diamonds\\
 2$\diamondsuit$ & 10+, 5+ clubs\\
 \hline
\end{tabular}
\end{center}
 Positive responses force to game. Opener can...
 \begin{itemize}
     \item \textbf{begin a relay} to find controls
     \item \textbf{bid a new suit} to ask a \textit{support question}
     \item \textbf{bid responder's suit} to ask for \textit{trump quality}
     \item \textbf{jump} to show a solid suit and ask for controls.
 \end{itemize}

 \subsection*{Asking bids}

 \subsection*{Dealing with interference}
 Opponents always interfere over a strong club. Always.

 \subsubsection*{Opponents overcall}
Passing shows 0--8 points.

Double shows 9+ and means that responder would have bid what the opponents just overcalled ("stolen bid").

Other responses are transfers.

If opponents overcall 1$\diamondsuit$, passing shows 0--5, doubling shows 6--8, and higher bids are unchanged.

\subsubsection*{Opponents double}
Pass shows 0--7, redoubling shows 8--9, 1$\diamondsuit$ shows 10+ and 5+ clubs, while higher bids are unchanged.

\subsubsection*{Responder is weak}
If responder shows weakness, opener can double for takeout, bid 1NT with a balanced 18--20, bid a new suit naturally with 15--20, or show 21+ and ask for support with a jump shift.

\subsubsection*{Responder is strong (10+)}
If advancer passes, opener can double for takeout, bid 1NT with a balanced 18--20, bid a new suit naturally with 15--20, or show 21+ and ask for support with a jump shift.






 \subsection*{1$\clubsuit$ in 3rd/4th seat}





\section{1$\diamondsuit\heartsuit\spadesuit$ opening}


\emph{\textbf{1st/2nd:} 10--14 HCP, 4+ cards \\ \textbf{3rd seat:} 8--14 HCP, 4+ cards \\ \textbf{4th seat:} 12--14 HCP, 4+ cards \\ \\
1$\diamondsuit$ may also show exactly 3334 shape.}

\subsection*{Responder's first bids}

Without interference, responder can bid a general response:
\begin{itemize}
    \item 1-over-1 to show 10--14 and 4+ cards
    \item 2-over-1 to show 13--14 and 4+ cards \textit{except 2$\heartsuit$ over 1$\spadesuit$ shows 5+ hearts}
    \item Negative 1NT to show 10--12 and a balanced hand or 7--10 and 3 care support for opener
\end{itemize}

In addition, responder has the following special responses available.

\subsubsection*{Blocking raises}

With 4-card support and 0--6 pts, raise to the two-level.

With 4-card support and 7--10 pts, double raise to the three-level.

If slam is unlikely, raise 1$\heartsuit\spadesuit$ to game.

\subsubsection*{Jump shifts}
Jumps in a new suit show weak hands with disinterest in the opening suit.

\subsubsection*{Key raise (Tangenthöjning)}
Bidding the suit under the opening bid at the two-level shows:
\begin{itemize}
    \item 4+ card support for opener and 10--12, or
    \item a natural suit and 13+ points
\end{itemize}

With a minimum, opener bids two of his original suit. Responder passes if he has trump support, or bids again if he had a two-over-one.

Otherwise opener invites game by bidding anything else.

\subsubsection*{Invitational 2NT}
2NT shows 5-card support and 13--17 points. Responder has a strong invitation, but is not interested in slam.

Opener bids three of his suit with a minimum.

\subsubsection*{3NT (pass or correct)}
15--17 points, no four-card major, no five-card minor, and exactly 3-card support.

\subsection*{Relays and weak takeouts}

Responder can begin a strong relay when holding 15+ points.

Responder can make a weak takeout of opener's suit with 443+ in the other suits and 0--9 points.
% 1$\heartsuit$ in response to 1$\diamondsuit$, or 1$\spadesuit$ to 1$\heartsuit$ is:

 1$\diamondsuit$--1$\heartsuit$ or 1$\heartsuit$--1$\spadesuit$ is:
\begin{itemize}
\item a natural 1-over-1, or
    \item a strong relay, or
        \item a weak takeout.
\end{itemize}

% $2\clubsuit$ in response to 1$\spadesuit$ is:
1$\spadesuit$--2$\clubsuit$ is:
\begin{itemize}
\item a natural 2-over-1, or
    \item a strong relay

\end{itemize}

% 1NT after opener's 1$\spadesuit$ is:
1$\spadesuit$--1NT is:
\begin{itemize}
\item a negative 1NT, or

        \item a weak takeout.
\end{itemize}

After any of these sequences, opener must bid again (except when balanced and minimum after 1$\spadesuit$--1NT) at the lowest level.

He
\begin{itemize}
    \item raises responder with
four-card support
\item bids NT if balanced,
\item bids a new
suit with 45, and
\item re-bids the opening suit
with 6 cards.
\end{itemize}

Responder's second action depends on what he has.

\subsubsection*{Responder began a strong relay}
If responder has 15+ he bids the next step (skipping 2 of opener's first suit), asking for top controls,
followed by question bidding.

Opener's third bid shows
number of peak controls and then the responder can
make support questions and trump questions in the same way
as in the questioning after 1c.

Sometimes the opening hand responds to the other
relay message at level 3h or higher, and then the
the support or drum call of the response hand is made at the four-position
and cannot be distinguished from a final bid.
If the opener's response to
peak control question is 3h or higher.
This can
be solved with a stop relay.
If the responder bids relay
(except 3N), the opener must also bid relay,
and after this the responder bids the final bid.

Also
a direct 3N from the responder is a final bid. If
the responder does not bid 3N or relay but a suit
this is a support or trump question.

\subsubsection*{Responder made a weak takeout}
If responder made a weak takeout of opener's suit, responder must not make a second bid.

Sometimes you have a choice between a weak relay and
raising with three-card support. Then
consider whether the opener might have a better
five-card suit on the side, when a weak
relay is often better than raising.

\subsubsection*{Responder made a natural 1-/2-over-1 or negative 1NT}
If responder's first bid was natural, he bids anything else. Bidding continues as if you had never heard of relays and takeouts.


\subsection*{Opener's second bid}
Assuming that responder made a general response (and not a special response):
\begin{itemize}
    \item Bids at lowest level show 10--12
    \item Jumps/reverses show 13--14
    \item NT shows a balanced hand
\end{itemize}

Rebidding the opening bid normally shows 6+ cards, bidding a new suit shows 54 shape.

Opener can raise responder's suit with 3+ card support.

With 5332, opener usually has to choose between bidding NT and rebidding the opening suit. At the 1-level, usually prefer NT, and at the 2-level, usually prefer rebidding.

Even with a maximum, opener should invite game after a 1-over-1 or negative 1NT, never blast it.

\subsection*{Responder's second bid}

\subsubsection*{4th suit forcing}
Shows invitational strength and asks for \begin{itemize}
    \item extra length in the suits bid so far
    \item 3+ card support for responder
    \item a stopper in the 4th suit
\end{itemize}

A jump in the 4th suit shows invitational strength and 55+ shape.

\subsubsection*{Weak jumping removal}
The jump of the response hand in the new suit shows a weak
hand of 0-9 points and at least six suits, and
a maximum of one card in the opening suit.


\subsubsection*{Higher response bids}
The higher response bids of the response hand are all weak and
based on a long suit. The opener has
normally only have to pass or possibly raise
the latch with good alignment.

\subsection*{Interference over 1$\diamondsuit\heartsuit\spadesuit$}
Opponents will often compete over 1$\diamondsuit\heartsuit\spadesuit$ by doubling or overcalling.

\subsubsection*{Opponents double (takeout)}

\begin{center}
\begin{tabular}{ |c|l| }
 \hline
 \multicolumn{2}{|c|}{\textit{Responses after 1x (X)}} \\
 \hline
 pass & balanced\\
 2/3/4\textit{x} &  blocking\\
 2NT & 13+, support showing\\
 suit & 10--14, natural, NF\\
 XX & 15+, strong relay\\
 \hline
\end{tabular}
\end{center}

With a strong balanced hand responder can pass and hope to later double for penalty.

\subsubsection*{Opponents overcall}

\begin{center}
\begin{tabular}{ |c|l| }
 \hline
 \multicolumn{2}{|c|}{\textit{Responses after 1x (1y)}} \\
 \hline
 X & 10--14, 44+ in unbid\\
  & 15+, 5+ in some suit\\
 raise &  blocking\\
 new suit & 10--14, 5+ cards\\
 1NT & 10--12, bal with stop \\
 2NT & 13+, 5+ card support\\
 3NT & 15--17, bal with stop \\
 cuebid & 13+, 4 card support \\
     & 15+, bal \textit{without} stop\\
 \hline
\end{tabular}
\end{center}

Jumps in a new suit show support for opener and a natural suit.

Doubling and then bidding a suit shows a stronger hand than bidding a suit directly.


\subsection*{Cuebids}
If we have bid a suit, a cuebid usually shows support and an invitational+ hand.

If we have bid several suits, the cuebid invites 3N, asking for a stopper.

Immediately after an opponent's 1-of-a-suit opening, a cuebid shows 10+ and 55 in the two highest unbid suits.

Additionally, you can always cuebid to show a strong hand and ask your partner to describe his hand further.

Double followed by a cuebid is used to
show a natural suit, if the opposing opening could be short.

Repeated cuebidding is also natural.

\section{1NT opening}

\normalfont
Use your favorite system. I'm serious: almost any system not masochistic works well after 1NT.

Personally, I use 12--14 HCP (no 5-card Major) in 1st/2nd seat and 15--17 HCP in 3rd/4th, but you can stick to one range throughout or use your own notrump ranges.


% (That said, I'd recommend using 2NT/3$\clubsuit$ as transfers to 3$\clubsuit$/3$\diamondsuit$. I'm still unhappy with 2$\spadesuit$ and 3$\diamondsuit\heartsuit\spadesuit$.)

\section{2$\clubsuit$ opening}


\emph{\textbf{1st/2nd:} 10--14 HCP, 5+ clubs \\
\textbf{3rd seat:} 8--14 HCP, 5+ clubs \\
\textbf{4th seat:} 12--14 HCP, 5+ clubs}

\begin{center}
\begin{tabular}{ |c|l| }
 \hline
 \multicolumn{2}{|c|}{\textit{Responses to 2$\clubsuit$}} \\
 \hline
 2$\diamondsuit$ & 13+, relay\\
 2$\heartsuit\spadesuit$ & 10--12, 5+ cards\\
 2NT &  13--14, balanced\\
3/4$\clubsuit$ & blocking\\
 3$\diamondsuit\heartsuit\spadesuit$ & 13--14, 5+ cards \\
 3NT & 15--17, sign-off \\
 \hline
\end{tabular}
\end{center}

Responder can use 2$\diamondsuit$ to learn more about opener.
\begin{center}
\begin{tabular}{ |c|l| }
 \hline
 \multicolumn{2}{|c|}{\textit{Opener after 2$\clubsuit$--2$\diamondsuit$}} \\
 \hline
 2$\heartsuit\spadesuit$ & 10--12, 4 cards\\
 2NT &  13--14, some 5332 \\
3$\clubsuit$ & 10--12, 6+ clubs\\
 3$\diamondsuit\heartsuit\spadesuit$ & 13--14, 4 cards \\
 3NT & 13--14, strong 6+ $\clubsuit$ \\
 \hline
\end{tabular}
\end{center}

If opponents interfere after 2$\clubsuit$ use the same methods as with interference after 1-of-a-suit.

\section{2$\diamondsuit\heartsuit\spadesuit$ opening}

\emph{\textbf{1st/2nd:} 0--8 HCP, 5 or 6 cards \\ \textbf{3rd seat:} 0--14 HCP, 5 or 6 cards \\ \textbf{4th seat:} 12--14 HCP, 6 cards}

\begin{center}
\begin{tabular}{ |c|l| }
 \hline
 \multicolumn{2}{|c|}{\textit{Responses to a weak-two}} \\
 \hline
 raise & sign-off\\
 new suit & 6+ card suit, NF  \\
2NT & 17+, forcing \\
 jump shift & sign-off \\
 3NT & sign-off \\
 \hline
\end{tabular}
\end{center}

2NT is the strong response:
\begin{center}
\begin{tabular}{ |c|l| }
 \hline
 \multicolumn{2}{|c|}{\textit{2$\diamondsuit$--2NT}} \\
 \hline
3$\clubsuit$ & 0--7, all hands\\
3$\diamondsuit$ & 8--9, no 4 card major \\
3$\heartsuit$ & 8--9, 4 hearts \\
3$\spadesuit$ & 8--9, 4 spades \\
 3NT & 44 in the majors \\
 \hline
\end{tabular}
\end{center}

\begin{center}
\begin{tabular}{ |c|l| }
 \hline
 \multicolumn{2}{|c|}{\textit{2$\heartsuit$--2NT}} \\
 \hline
3$\clubsuit$ & 0--7, all hands\\
3$\diamondsuit$ & 8--9, weak hearts \\
3$\heartsuit$ & 8--9, strong hearts \\
3$\spadesuit$ & 8--9, weak $\heartsuit$, 4 $\spadesuit$ \\
 3NT & 8--9, strong $\heartsuit$, 4 $\spadesuit$ \\
 \hline
\end{tabular}
\end{center}

\begin{center}
\begin{tabular}{ |c|l| }
 \hline
 \multicolumn{2}{|c|}{\textit{2$\spadesuit$--2NT}} \\
 \hline
3$\clubsuit$ & 0--7, all hands\\
3$\diamondsuit$ & 8--9, weak spades \\
3$\heartsuit$ & 8--9, weak $\spadesuit$, 4 $\heartsuit$ \\
3$\spadesuit$ & 8--9, strong spades \\
 3NT & 8--9, strong $\spadesuit$, 4 $\heartsuit$ \\
 \hline
\end{tabular}
\end{center}


\section{Higher openings}

2NT shows a balanced 21--22 HCP.

3NT shows a 7+ card solid major suit.

Other suit preempts are natural and follow the rule of 2-3-4. Please experiment with your partner.



\section{Notes and Choices}
% \itshape

\subsubsection*{Canape}
 Kaniklöver also works with canape! You'd use canape on any $\diamondsuit\heartsuit/ \diamondsuit\spadesuit/ \heartsuit\spadesuit$ two-suited hand (everything else is still ``open your longest suit''). You can call some 5332 hands two-suited, but be ready to play quite a few 3-3 fits at the 2-level!



    % \subsubsection*{Run-out system after 1NT (X)}
    % Everyone who plays a weak NT devises their own run-out system: here's mine!


    % \begin{center}
    % \begin{tabular}{ |c|l| }
    %  \hline
    %  \multicolumn{2}{|c|}{\textit{Responses after a penalty double}} \\
    %  \hline
    %  pass & Relay to XX \\
    %  XX & Relay to 2$\clubsuit$ \\
    %  2$\clubsuit$ & Relay to 2$\diamondsuit$ \\
    %  2$\diamondsuit\heartsuit$ & Transfers to 2$\heartsuit\spadesuit$ \\
    %  2NT & Strong and unbalanced\\
    %  \hline
    % \end{tabular}
    % \end{center}

    % % With a strong hand or no fiveWith a one-suited hand, relay/transfer.


    % \begin{center}
    % \begin{tabular}{ |c|l| }
    %  \hline
    %  \multicolumn{2}{|c|}{\textit{After 1NT (X) -- -- XX --}} \\
    %  \hline
    %  pass & 8+ HCP (penalty XX) \\
    %  2$\clubsuit$ & (3 or) 4 clubs \\
    %  2$\diamondsuit$ & 44 in diamonds and hearts \\
    %  2$\heartsuit$ & 44 in the majors \\
    %  \hline
    % \end{tabular}
    % \end{center}

    % \begin{center}
    % \begin{tabular}{ |c|l| }
    %  \hline
    %  \multicolumn{2}{|c|}{\textit{After 1NT (X) XX -- 2$\clubsuit$ --}} \\
    %  \hline
    %  pass & 5+ clubs \\
    %  2$\diamondsuit$ & 44 in diamonds + spades \\
    %  2$\heartsuit$ & 44 in the majors \\
    %  \hline
    % \end{tabular}
    % \end{center}

    % \begin{center}
    % \begin{tabular}{ |c|l| }
    %  \hline
    %  \multicolumn{2}{|c|}{\textit{After 1NT (X) 2$\clubsuit$ -- 2$\diamondsuit$ --}} \\
    %  \hline
    %  pass & 5+ diamonds \\
    %  2$\heartsuit$ & 44 in the majors \\
    %  \hline
    % \end{tabular}
    % \end{center}

\subsubsection*{Run-out system after 1NT (X)}
Here's a system I devised: it emphasises having the 1NT opener declare whenever responder has shape.

\begin{center}
\begin{tabular}{ |c|l| }
 \hline
 \multicolumn{2}{|c|}{\textit{Responses after 1NT (X)}} \\
 \hline
 pass & Relay to XX \\
 XX & Relay to 2$\clubsuit$ \\
 2$\clubsuit\diamondsuit\heartsuit$ & Transfers to 2$\diamondsuit\heartsuit\spadesuit$ \\
 2NT & Strong, unbalanced\\
 \hline
\end{tabular}
\end{center}

% With a strong hand or no fiveWith a one-suited hand, relay/transfer.


\begin{center}
\begin{tabular}{ |c|l| }
 \hline
 \multicolumn{2}{|c|}{\textit{Resp. after 1NT (X) -- -- XX --}} \\
 \hline
 pass & Happy with 1NTxx \\
 2$\clubsuit$ & 3 or 4 clubs, no 5cM \\
 2$\diamondsuit$ & 443+ outside clubs \\
 \hline
\end{tabular}
\end{center}

\begin{center}
\begin{tabular}{ |c|l| }
 \hline
 \multicolumn{2}{|c|}{\textit{Resp. after 1NT (X) XX -- 2$\clubsuit$ --}} \\
 \hline
 pass & 5+ clubs \\
 2$\diamondsuit$ & 44+ in the majors \\

 \hline
\end{tabular}
\end{center}

Of course, the double might instead come from balancing seat. Opener must then pass.

\begin{center}
\begin{tabular}{ |c|l| }
 \hline
 \multicolumn{2}{|c|}{\textit{Resps. after 1NT -- -- (X) -- --}} \\
 \hline
 pass & Happy with 1NTx \\
 XX & 5+ card minor \\
 2$\clubsuit$ & 3 or 4 clubs \\
 2$\diamondsuit$ &  443+ outside clubs \\
 \hline
\end{tabular}
\end{center}


\subsubsection*{Sounder opening bids}
You can make opener's bids all require one point more (or two!) if you want, but then subtract one point from each of responder's replies.


\subsection*{Statistics}

\subsubsection*{How many cards does each opening bid show?}
%\large\sffamily
\begin{center}
\begin{tabular}{ |r|ccccc| }
\hline
  & 3 & 4 & 5 & 6 & 7+ \\
 \hline
& \multicolumn{5}{|c|}{\textit{Kaninklöver 12--14 NT}} \\
1$\diamondsuit$ & 7 & 16 & 48 & 24 & 6 \\
1$\heartsuit$ & & 32 & 47 & 17 & 4 \\
 1$\spadesuit$ & & 24 & 52 & 19 & 5 \\
2$\clubsuit$ & & & 60 & 32 & 8 \\
 \hline
 & \multicolumn{5}{|c|}{\textit{Kaninklöver 15--17 NT}} \\
1$\diamondsuit$ & 10 & 25 & 44 & 17 & 5 \\
1$\heartsuit$ & & 47 & 37 & 13 & 3 \\
 1$\spadesuit$ & & 38 & 42 & 16 & 4 \\
2$\clubsuit$ & & & 66 & 27 & 7 \\
 \hline
 & \multicolumn{5}{|c|}{\textit{Standard American}} \\
1$\clubsuit$ & 12 & 21 & 39 & 22 & 6 \\
1$\diamondsuit$ & 4 & 37 & 37 & 18 & 4 \\
1$\heartsuit$ & & & 65 & 28 & 7 \\
1$\spadesuit$ & & & 66 & 27 & 7  \\
\hline
  & 3 & 4 & 5 & 6 & 7+ \\
 \hline
\end{tabular}
\end{center}

% \begin{center}
% \begin{tabular}{ |cc|cc|cc| }
%  \hline
%  1$\clubsuit$   & 12 & 12$\clubsuit$  & gh    & 3x & \\
% 1$\diamondsuit$ & 10 & 2$\diamondsuit$ & 37   & 3NT &  5 \\
% 1$\heartsuit$   &   & 2$\heartsuit$ & 3x        & 4x & \\
% 1$\spadesuit$   &   & 2$\spadesuit$ & 38        &  &  \\
% \hline
% \end{tabular}
% \end{center}

% \begin{figure}
%     \centering
%     \includegraphics{Pictures/teste}
%     \caption{Caption}
%     \label{fig:my_label}
% \end{figure}

Probabilities in percentages. For Standard American, I assume you open 1$\diamondsuit$ with equally long minors (except when 33).

\end{multicols}


\end{document}
